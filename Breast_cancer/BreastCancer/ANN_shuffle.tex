\documentclass[]{article}
\usepackage{lmodern}
\usepackage{amssymb,amsmath}
\usepackage{ifxetex,ifluatex}
\usepackage{fixltx2e} % provides \textsubscript
\ifnum 0\ifxetex 1\fi\ifluatex 1\fi=0 % if pdftex
  \usepackage[T1]{fontenc}
  \usepackage[utf8]{inputenc}
\else % if luatex or xelatex
  \ifxetex
    \usepackage{mathspec}
  \else
    \usepackage{fontspec}
  \fi
  \defaultfontfeatures{Ligatures=TeX,Scale=MatchLowercase}
\fi
% use upquote if available, for straight quotes in verbatim environments
\IfFileExists{upquote.sty}{\usepackage{upquote}}{}
% use microtype if available
\IfFileExists{microtype.sty}{%
\usepackage{microtype}
\UseMicrotypeSet[protrusion]{basicmath} % disable protrusion for tt fonts
}{}
\usepackage[margin=1in]{geometry}
\usepackage{hyperref}
\hypersetup{unicode=true,
            pdftitle={ANN},
            pdfauthor={KWONHYUNJIN},
            pdfborder={0 0 0},
            breaklinks=true}
\urlstyle{same}  % don't use monospace font for urls
\usepackage{color}
\usepackage{fancyvrb}
\newcommand{\VerbBar}{|}
\newcommand{\VERB}{\Verb[commandchars=\\\{\}]}
\DefineVerbatimEnvironment{Highlighting}{Verbatim}{commandchars=\\\{\}}
% Add ',fontsize=\small' for more characters per line
\usepackage{framed}
\definecolor{shadecolor}{RGB}{248,248,248}
\newenvironment{Shaded}{\begin{snugshade}}{\end{snugshade}}
\newcommand{\AlertTok}[1]{\textcolor[rgb]{0.94,0.16,0.16}{#1}}
\newcommand{\AnnotationTok}[1]{\textcolor[rgb]{0.56,0.35,0.01}{\textbf{\textit{#1}}}}
\newcommand{\AttributeTok}[1]{\textcolor[rgb]{0.77,0.63,0.00}{#1}}
\newcommand{\BaseNTok}[1]{\textcolor[rgb]{0.00,0.00,0.81}{#1}}
\newcommand{\BuiltInTok}[1]{#1}
\newcommand{\CharTok}[1]{\textcolor[rgb]{0.31,0.60,0.02}{#1}}
\newcommand{\CommentTok}[1]{\textcolor[rgb]{0.56,0.35,0.01}{\textit{#1}}}
\newcommand{\CommentVarTok}[1]{\textcolor[rgb]{0.56,0.35,0.01}{\textbf{\textit{#1}}}}
\newcommand{\ConstantTok}[1]{\textcolor[rgb]{0.00,0.00,0.00}{#1}}
\newcommand{\ControlFlowTok}[1]{\textcolor[rgb]{0.13,0.29,0.53}{\textbf{#1}}}
\newcommand{\DataTypeTok}[1]{\textcolor[rgb]{0.13,0.29,0.53}{#1}}
\newcommand{\DecValTok}[1]{\textcolor[rgb]{0.00,0.00,0.81}{#1}}
\newcommand{\DocumentationTok}[1]{\textcolor[rgb]{0.56,0.35,0.01}{\textbf{\textit{#1}}}}
\newcommand{\ErrorTok}[1]{\textcolor[rgb]{0.64,0.00,0.00}{\textbf{#1}}}
\newcommand{\ExtensionTok}[1]{#1}
\newcommand{\FloatTok}[1]{\textcolor[rgb]{0.00,0.00,0.81}{#1}}
\newcommand{\FunctionTok}[1]{\textcolor[rgb]{0.00,0.00,0.00}{#1}}
\newcommand{\ImportTok}[1]{#1}
\newcommand{\InformationTok}[1]{\textcolor[rgb]{0.56,0.35,0.01}{\textbf{\textit{#1}}}}
\newcommand{\KeywordTok}[1]{\textcolor[rgb]{0.13,0.29,0.53}{\textbf{#1}}}
\newcommand{\NormalTok}[1]{#1}
\newcommand{\OperatorTok}[1]{\textcolor[rgb]{0.81,0.36,0.00}{\textbf{#1}}}
\newcommand{\OtherTok}[1]{\textcolor[rgb]{0.56,0.35,0.01}{#1}}
\newcommand{\PreprocessorTok}[1]{\textcolor[rgb]{0.56,0.35,0.01}{\textit{#1}}}
\newcommand{\RegionMarkerTok}[1]{#1}
\newcommand{\SpecialCharTok}[1]{\textcolor[rgb]{0.00,0.00,0.00}{#1}}
\newcommand{\SpecialStringTok}[1]{\textcolor[rgb]{0.31,0.60,0.02}{#1}}
\newcommand{\StringTok}[1]{\textcolor[rgb]{0.31,0.60,0.02}{#1}}
\newcommand{\VariableTok}[1]{\textcolor[rgb]{0.00,0.00,0.00}{#1}}
\newcommand{\VerbatimStringTok}[1]{\textcolor[rgb]{0.31,0.60,0.02}{#1}}
\newcommand{\WarningTok}[1]{\textcolor[rgb]{0.56,0.35,0.01}{\textbf{\textit{#1}}}}
\usepackage{graphicx,grffile}
\makeatletter
\def\maxwidth{\ifdim\Gin@nat@width>\linewidth\linewidth\else\Gin@nat@width\fi}
\def\maxheight{\ifdim\Gin@nat@height>\textheight\textheight\else\Gin@nat@height\fi}
\makeatother
% Scale images if necessary, so that they will not overflow the page
% margins by default, and it is still possible to overwrite the defaults
% using explicit options in \includegraphics[width, height, ...]{}
\setkeys{Gin}{width=\maxwidth,height=\maxheight,keepaspectratio}
\IfFileExists{parskip.sty}{%
\usepackage{parskip}
}{% else
\setlength{\parindent}{0pt}
\setlength{\parskip}{6pt plus 2pt minus 1pt}
}
\setlength{\emergencystretch}{3em}  % prevent overfull lines
\providecommand{\tightlist}{%
  \setlength{\itemsep}{0pt}\setlength{\parskip}{0pt}}
\setcounter{secnumdepth}{0}
% Redefines (sub)paragraphs to behave more like sections
\ifx\paragraph\undefined\else
\let\oldparagraph\paragraph
\renewcommand{\paragraph}[1]{\oldparagraph{#1}\mbox{}}
\fi
\ifx\subparagraph\undefined\else
\let\oldsubparagraph\subparagraph
\renewcommand{\subparagraph}[1]{\oldsubparagraph{#1}\mbox{}}
\fi

%%% Use protect on footnotes to avoid problems with footnotes in titles
\let\rmarkdownfootnote\footnote%
\def\footnote{\protect\rmarkdownfootnote}

%%% Change title format to be more compact
\usepackage{titling}

% Create subtitle command for use in maketitle
\newcommand{\subtitle}[1]{
  \posttitle{
    \begin{center}\large#1\end{center}
    }
}

\setlength{\droptitle}{-2em}
  \title{ANN}
  \pretitle{\vspace{\droptitle}\centering\huge}
  \posttitle{\par}
  \author{KWONHYUNJIN}
  \preauthor{\centering\large\emph}
  \postauthor{\par}
  \predate{\centering\large\emph}
  \postdate{\par}
  \date{2018뀈 9썡 20씪}


\begin{document}
\maketitle

\#nnet: ANN 패키지

\begin{Shaded}
\begin{Highlighting}[]
\NormalTok{bc <-}\StringTok{ }\KeywordTok{read.xlsx}\NormalTok{(}\StringTok{"New_version_breast_cancer.xlsx"}\NormalTok{,}\DecValTok{1}\NormalTok{)}
\KeywordTok{head}\NormalTok{(bc)}
\end{Highlighting}
\end{Shaded}

\begin{verbatim}
##   NA. age tumor.size inv.nodes deg.malig irradiat   Class lt40 ge40
## 1   1   4          4         1         3        0   recur    0    0
## 2   2   5          4         1         1        0 norecur    0    1
## 3   3   5          8         1         2        0   recur    0    1
## 4   4   4          8         1         3        1 norecur    0    0
## 5   5   4          7         2         2        0   recur    0    0
## 6   6   5          6         2         2        1 norecur    0    0
##   premeno node.capse no.node.capse breast.left breast.right quad.cen
## 1       1          1             0           0            1        0
## 2       0          0             1           0            1        1
## 3       0          0             1           1            0        0
## 4       1          1             0           0            1        0
## 5       1          1             0           1            0        0
## 6       1          0             1           0            1        0
##   quad.Ll quad.Lu quad.Rl quad.Ru
## 1       0       1       0       0
## 2       0       0       0       0
## 3       1       0       0       0
## 4       1       0       0       0
## 5       0       0       0       1
## 6       0       1       0       0
\end{verbatim}

\#현재 bc 데이터 프레임에 id라는 컬럼이 필요없다. \#bc\_shuffle에서
1번컬럼은 제외하고 나머지 컬럼을 bc2에 할당한다.

\begin{Shaded}
\begin{Highlighting}[]
\NormalTok{ideal <-}\StringTok{ }\KeywordTok{class.ind}\NormalTok{(bc}\OperatorTok{$}\NormalTok{Class)}
\NormalTok{bc2 <-bc[}\OperatorTok{-}\DecValTok{1}\NormalTok{]}
\KeywordTok{head}\NormalTok{(bc2)}
\end{Highlighting}
\end{Shaded}

\begin{verbatim}
##   age tumor.size inv.nodes deg.malig irradiat   Class lt40 ge40 premeno
## 1   4          4         1         3        0   recur    0    0       1
## 2   5          4         1         1        0 norecur    0    1       0
## 3   5          8         1         2        0   recur    0    1       0
## 4   4          8         1         3        1 norecur    0    0       1
## 5   4          7         2         2        0   recur    0    0       1
## 6   5          6         2         2        1 norecur    0    0       1
##   node.capse no.node.capse breast.left breast.right quad.cen quad.Ll
## 1          1             0           0            1        0       0
## 2          0             1           0            1        1       0
## 3          0             1           1            0        0       1
## 4          1             0           0            1        0       1
## 5          1             0           1            0        0       0
## 6          0             1           0            1        0       0
##   quad.Lu quad.Rl quad.Ru
## 1       1       0       0
## 2       0       0       0
## 3       0       0       0
## 4       0       0       0
## 5       0       0       1
## 6       1       0       0
\end{verbatim}

\#Train 과 Test 를 9:1로 나눈다

\begin{Shaded}
\begin{Highlighting}[]
\NormalTok{train_num<-}\KeywordTok{round}\NormalTok{(}\FloatTok{0.9}\OperatorTok{*}\KeywordTok{nrow}\NormalTok{(bc2),}\DecValTok{0}\NormalTok{)}
\NormalTok{bc_train<-bc2[}\DecValTok{1}\OperatorTok{:}\NormalTok{train_num,]}
\NormalTok{bc_test<-bc2[(train_num}\OperatorTok{+}\DecValTok{1}\NormalTok{)}\OperatorTok{:}\KeywordTok{nrow}\NormalTok{(bc2),]}
\end{Highlighting}
\end{Shaded}

\#size = 2는 hidden layer가 2층 이라는 뜻 \#decay = 5e-04는
overfitting을 피하기 위해 사용하는 weight decay parametem, default 값은
0 \#maxit = 200은 200번 반복

\begin{Shaded}
\begin{Highlighting}[]
\NormalTok{model <-}\StringTok{ }\KeywordTok{nnet}\NormalTok{(Class }\OperatorTok{~}\StringTok{ }\NormalTok{., }\DataTypeTok{data =}\NormalTok{ bc_train, }\DataTypeTok{size =} \DecValTok{2}\NormalTok{, }\DataTypeTok{decay =} \FloatTok{5e-04}\NormalTok{, }\DataTypeTok{maxit =} \DecValTok{50}\NormalTok{)}
\end{Highlighting}
\end{Shaded}

\begin{verbatim}
## # weights:  39
## initial  value 164.525641 
## iter  10 value 135.993964
## iter  20 value 122.720332
## iter  30 value 116.318940
## iter  40 value 115.973936
## iter  50 value 115.771334
## final  value 115.771334 
## stopped after 50 iterations
\end{verbatim}

\#분류를 얻을때는 type=``class''를 지정해야하지만, 기본 값이 class이므로
생략 가능

\begin{Shaded}
\begin{Highlighting}[]
\NormalTok{pred <-}\StringTok{ }\KeywordTok{predict}\NormalTok{(model,bc_test,}\DataTypeTok{type=}\StringTok{"class"}\NormalTok{)}
\KeywordTok{table}\NormalTok{(pred,bc_test}\OperatorTok{$}\NormalTok{Class)}
\end{Highlighting}
\end{Shaded}

\begin{verbatim}
##          
## pred      norecur recur
##   norecur      21     5
##   recur         1     1
\end{verbatim}

\begin{Shaded}
\begin{Highlighting}[]
\KeywordTok{confusionMatrix}\NormalTok{(}\KeywordTok{table}\NormalTok{(pred, bc_test}\OperatorTok{$}\NormalTok{Class))}
\end{Highlighting}
\end{Shaded}

\begin{verbatim}
## Confusion Matrix and Statistics
## 
##          
## pred      norecur recur
##   norecur      21     5
##   recur         1     1
##                                          
##                Accuracy : 0.7857         
##                  95% CI : (0.5905, 0.917)
##     No Information Rate : 0.7857         
##     P-Value [Acc > NIR] : 0.6071         
##                                          
##                   Kappa : 0.16           
##  Mcnemar's Test P-Value : 0.2207         
##                                          
##             Sensitivity : 0.9545         
##             Specificity : 0.1667         
##          Pos Pred Value : 0.8077         
##          Neg Pred Value : 0.5000         
##              Prevalence : 0.7857         
##          Detection Rate : 0.7500         
##    Detection Prevalence : 0.9286         
##       Balanced Accuracy : 0.5606         
##                                          
##        'Positive' Class : norecur        
## 
\end{verbatim}


\end{document}
